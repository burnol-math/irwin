% -*- time-stamp-active: nil -*-
\def\timestamp{Time-stamp: <2025-05-16 12:28:30 CEST>}
\def\timestampaux Time-stamp: <#1 CE#2>\relax{#1}
\edef\timestamp{\expandafter\timestampaux\timestamp\relax}

\documentclass[a4paper,french,dvipdfmx,11pt]{article}
\usepackage[T1]{fontenc}
\usepackage[english]{babel}

\usepackage[np,autolanguage]{numprint}
\usepackage{amsmath,amsthm,amssymb}
\usepackage{mathtools}

\usepackage[hscale=0.666,vscale=0.7]{geometry}

\DeclareFontFamily{OMX}{mlmex}{}
\DeclareFontShape{OMX}{mlmex}{m}{n}{%
   <->mlmex10%
   }{}%
\usepackage{mlmodern}

\usepackage{hyperref}
\hypersetup{%
  breaklinks=true,%
  colorlinks=true,%
%  linkcolor={blue},%
%  citecolor={red},%
  pdfauthor={Jean-François Burnol},%
%  pdfsubject={},%
  pdfstartview=FitH,%
  pdfpagemode=UseNone,%
}
\usepackage{hypcap}
\usepackage{enumitem}
\usepackage{ragged2e}
\begin{document}

\begin{center}
  \Large \texttt{irwin\_v5}: choice of \texttt{Mmax}
\\
\smallskip
\normalsize
\texttt{J.-F. Burnol \timestamp}\par
\end{center}

For the mathematical justification of the actual algorithms see the articles
linked-to at the
\href{https://gitlab.com/burnolmath/irwin/-/blob/master/README.md}{README.md}.
The aim here is to document the choice of \texttt{Mmax}.  The explanations
given were formerly inside the \texttt{irwin\_v5.sage} as a somewhat
lengthy comment.  We format it using \TeX{} math notations.

We estimate how many terms are needed according to \texttt{level} $l$ and the
targeted final precision (before rounding away the guard bits)
\texttt{nbbits}. This estimate relies on two things:
\begin{itemize}[noitemsep]
\item a priori lower bound for the final result,
\item a priori upper bounds for the terms of the series.
\end{itemize}

\textcolor{red}{TODO}: revisit explanations next and make them clearer.

\linespread{1.2}\selectfont

The $u_{k;m}$'s are bounded above by $b$. In fact they are bounded above by
$\frac{b}{m+1}$ but we don't have this upper bound for the $v_{k;m}$'s.  For
the latter the beta's are (a bit) smaller as they use inverse powers of some $n+1$'s
not $n$'s.

We always take $l=\mathtt{level}$ at least $2$.  The crucial upper bound for
the $m$\textsuperscript{th} term of the series comes from the $\beta_{m+1}$'s
(\textbf{ATTENTION}: these $\beta_{m+1}$'s also depend on some number of
occurrences parameter, not explicited here in the notation).%
%
\marginpar{\linespread{1}\selectfont\RaggedRight
   \textcolor{red}{TODO}: fill-in the details and check.}
%
  We can estimate
that $\beta_{m+1}$ is bounded above
 by $\frac{b^{-(l-1)} +
  m^{-1}}{b^{(l-1)m}}$ with here $l$ being the level.  For $k=0$ and $d=1$, we
can improve that to $\frac{\frac12 b^{-(l-1)}+m^{-1}}{2^m\cdot b^{(l-1)m}}$.

We know from Farhi's theorem that $b\log(b)$ is a lower bound of total sum $S$
(i.e.\@ here $S$ refers not only to the Burnol series but includes the initial, main,
contributions) if $k>0$ or if $k=0$ and $d=0$.  If
$k=0$ and $d>0$ we have Proposition 6 in the Irwin paper which says $S >
b\log(b) -b\log(1+1/d)$. Worst case is with $d=1$, giving $b\log(b/2)$. For
$d$ at least $2$ the lower bound improves to $b\log(2b/3)$. Using only that
the $u_{k;m}$'s and $v_{k;m}$'s are bounded above by $b$, we will need to
compare $\log(b/2)$ with $((2b^{l-1})^{-1}+m^{-1})2^{-m}$, and $\log(2b/3)$
with $b^{-(l-1)} + m^{-1}$.  We always take $l$ at least $2$.  So it is
$\log(b/2)$ versus $(1/(2b)+1/m)/2^m$ or $\log(2b/3)$ versus $1/b + 1/m$.

Testing the worst case $m=1$, we have $\log(b/2)>(1/(2b)+1)/2$ starting at
$b=4$, and $\log(2b/3)>1/b + 1$ for $b$ at least $5$.

For $m=2$, we have $\log(b/2)>(1/(2b)+1/2)/4$ for $b$ at least $3$ and
$\log(2b/3)>1/b+1/2$ for $b$ at least $4$.  But for $b=3$, the difficulty with
$m=2$ is for $k=0$ and $d$ neither $0$ (as $\log(3)>1>1/3+1/2$) or $1$,
hence $d=2$. One finds numerically $S>2.682$, hence in that case
$S/3>0.894>1/3+1/2$ indeed.

Only remains to check for $b=2$.  The only Irwin sum not $>2\log(2)$ is with
$k=0$ and $d=1$. This is the empty sum with value zero.  We do not care too
much about our estimates then, use
\texttt{irwin(2,1,0)} or \texttt{irwinpos(2,1,0)} at your own risk... but it
seems to work fine.

For $\mathtt{level}=2$, $\beta_{m+1}$ for $b=2$ is bounded above by
$1/2^{m+1}+1/3^{m+1}$ and $u_{k;m}$ by $2/(m+1)$.  We want to compare this
with $2 \log(2)/2^m$. It is less already for $m=1$.

For the positive series we have $2(1/3^{m+1}+1/4^{m+1})$ to compare with
$2\log(2)/2^m$. Again it is less already for $m=1$.

For $\mathtt{level}>2$ and the alternating series a more precise upper bound
of the $m$\textsuperscript{th} term is $(1/b^{l-1}+1/m)\frac{b}{m+1}$ times
$b^{-(l-1)m}$ so we compare $(1/4+1/m)/(m+1)$ with $\log(2)$.  And already for
$m=1$ it is smaller. Still for $b=2$ and $l>2$, for the positive series we can
bound above the $m$\textsuperscript{th} term by
$(1/(b^{l-1}+1)+1/m)\cdot(b^{l-1}/(b^{l-1}+1))^m b$ times $b^{-(l-1}m)$ using
only $v_{k;m}<= b$. And we need to compare with $2 \log(2)$. So we check if
$(1/5+1/m)(4/5)^m$ is less than $\log(2)$. This is true for $m$ at least $2$.

In conclusion it is always true that for $m$ at least $2$ the
$m$\textsuperscript{th} term of the series contributes a fraction less than
$b^{-(l-1)m}$ of the Kempner-Irwin value (we could make a detailed examination
for $m=1$, but drop it). For the alternative series this gives a bound for the
total error made by neglecting all terms starting with the
$m$\textsuperscript{th} one; for the positive series, we have to take into
account an extra factor of $b^{l-1}/(b^{l-1}-1)$ which is at most $2$.  We
don't worry about this $2$.

So, we only need to take into account the $m$\textsuperscript{th} term of the
series with a precision equal to $\mathtt{nbbits} - (l-1)m\log(b,2)$ where
\texttt{nbbits} is the ``full'' precision (inclusive of guard bits).  We
choose the number of terms \texttt{Mmax} such that it is the largest integer
such that $\mathtt{nbbits} - (l-1)\cdot\mathtt{Mmax}\cdot\log(b,2)\geq
\mathtt{nbguardbits}/2$.

Hence it is defined as (using notation \verb+_Mmax+ in case user wants to pass
custom choice as \texttt{Mmax}):
\begin{verbatim}
    _Mmax = floor((nbbits - nbguardbits/2)/(level-1)/log(b,2))
\end{verbatim}

We use precisions of the type $\mathtt{nbbits} - j T$ with $T =
\mathtt{PrecStep}$.  We will have the need only for $j$'s with $\mathtt{nbbits}
-jT \geq \mathtt{nbguardbits}/2$ and will thus use
\begin{verbatim}
    NbOfPrec = 1 + floor((nbbits - nbguardbits/2)/PrecStep)
\end{verbatim}
distinct precisions.  For the actual mapping of index $m$ to the integer $j$
see the code comments in \texttt{irwin\_v5.sage}.


\textbf{NOTA BENE:}

In the case $d=b-1$, there is an additional factor $((b-2)/(b-1))^m$ coming
from $u_{0;m}$ which is not taken into account here (we used only estimates
for $\beta_{m+1}$'s basically), so we end up using more terms than needed in
this case.  For example at $\mathtt{level} = 2$, computing the classic Kempner
sum for 500 digits will use about 500 terms but 475 would have been enough as
$(8/9)^{475}$ is about $\np{5e-25}$.  For $k>0$ and increasing, the effect
diminishes.  (One can use \texttt{verbose=True} to examine the size of the
smallest kept term).

               For the positive series, there is no such factor
               but the $\beta_{m+1}$ is smaller for $\mathtt{level}=2$ roughly
               by $(b/(b+1))^m$ (for $d=1$ this will be rather with
               $b/(b+1/2)$, and $d=0$ also has another ratio) so we
               have a similar phenomenon of using too many terms
               but for other reasons.

               For $\mathtt{level}=3$ (which is the default) effect will be
               lower than for $\mathtt{level}=2$.

               With $k=0$ (only) and excluded digit 1, we use way
               too many terms due to roughly a $2^{-m}$ factor not
               being taken into account.  User can set \texttt{Mmax}
               explicitly to their own choosing.

               Here we have chosen \texttt{Mmax} according to an analysis
               which is supposed to work for all $k$'s and $d$'s
               and $\mathtt{level}>1$.

\end{document}
